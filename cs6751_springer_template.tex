
%%%%%%%%%%%%%%%%%%%%%%% file typeinst.tex %%%%%%%%%%%%%%%%%%%%%%%%%
%
% This is the LaTeX source for the instructions to authors using
% the LaTeX document class 'llncs.cls' for contributions to
% the Lecture Notes in Computer Sciences series.
% http://www.springer.com/lncs       Springer Heidelberg 2006/05/04
%
% It may be used as a template for your own input - copy it
% to a new file with a new name and use it as the basis
% for your article.
%
% NB: the document class 'llncs' has its own and detailed documentation, see
% ftp://ftp.springer.de/data/pubftp/pub/tex/latex/llncs/latex2e/llncsdoc.pdf
%
%%%%%%%%%%%%%%%%%%%%%%%%%%%%%%%%%%%%%%%%%%%%%%%%%%%%%%%%%%%%%%%%%%%


\documentclass[runningheads,letterpaper]{llncs}

\usepackage{amssymb}
\setcounter{tocdepth}{3}
\usepackage{graphicx}
\usepackage[]{epstopdf}
\usepackage{url}
\urldef{\mailsa}\path|{alfred.hofmann, ursula.barth, ingrid.haas, frank.holzwarth,|
\urldef{\mailsb}\path|anna.kramer, leonie.kunz, christine.reiss, nicole.sator,|
\urldef{\mailsc}\path|erika.siebert-cole, peter.strasser, lncs}@springer.com|    
\newcommand{\keywords}[1]{\par\addvspace\baselineskip
\noindent\keywordname\enspace\ignorespaces#1}

\begin{document}

\mainmatter  % start of an individual contribution

% first the title is needed
\title{\huge{Report for CS 6751, Spring 2017} \\ \Large{Grasping and Prehensile Manipulation Project}}

% a short form should be given in case it is too long for the running head
\titlerunning{Lecture Notes in Computer Science: Authors' Instructions}

% the name(s) of the author(s) follow(s) next
%
% NB: Chinese authors should write their first names(s) in front of
% their surnames. This ensures that the names appear correctly in
% the running heads and the author index.
%
\author{Vighnesh Vatsal}
%\thanks{Please note that the LNCS Editorial assumes that all authors have used
%the western naming convention, with given names preceding surnames. This determines
%the structure of the names in the running heads and the author index.}%
%\and Ursula Barth\and Ingrid Haas\and Frank Holzwarth\and\\
%Anna Kramer\and Leonie Kunz\and Christine Rei\ss\and\\
%Nicole Sator\and Erika Siebert-Cole\and Peter Stra\ss er}
%
\authorrunning{Lecture Notes in Computer Science: Authors' Instructions}
% (feature abused for this document to repeat the title also on left hand pages)

% the affiliations are given next; don't give your e-mail address
% unless you accept that it will be published
\institute{Sibley School of Mechanical and Aerospace Engineering\\
Cornell University}
%Tiergartenstr. 17, 69121 Heidelberg, Germany\\
%\mailsa\\
%\mailsb\\
%\mailsc\\
%\url{http://www.springer.com/lncs}}

%
% NB: a more complex sample for affiliations and the mapping to the
% corresponding authors can be found in the file "llncs.dem"
% (search for the string "\mainmatter" where a contribution starts).
% "llncs.dem" accompanies the document class "llncs.cls".
%

%\toctitle{Lecture Notes in Computer Science}
%\tocauthor{Authors' Instructions}
\maketitle


%\begin{abstract}
%The abstract should summarize the contents of the paper and should
%contain at least 70 and at most 150 words. It should be written using the
%\emph{abstract} environment.
%\keywords{We would like to encourage you to list your keywords within
%the abstract section}
%\end{abstract}

\section{Introduction}


\section{Literature Review}
Grasping end-effectors are one of the most commonly employed manipulation tools in robotic arms. Most literature on grasping-based manipulation~\cite{prattichizzo2016grasping} considers the scenario in which the robotic arm firmly grips the object, immobilizing it and making it a part of the arm. To place the object in its desired goal configuration is then considered to be a path-planning problem in the $C$--space of the robotic arm~\cite{brock2008motion}, which has been augmented with the grasped object. 

On the contrary, humans tend not to manipulate objects in their environment with only pick-and-place style grasping. For instance, to move a box across a table, one may slide it over the surface by pushing~\cite{chavan2015prehensile}. This form of motion is termed as \textit{prehensile} manipulation. In this class of motion, the object is not assumed to be an extension of the manipulator. Based on the geometry of the object and the robot's end-effector, we can identify configurations that either completely immobilize an object (grasps), or constrain it in such a way that it cannot escape to infinity, but is not completely immobilized (cages). The full geometric framework for these conditions is presented in~\cite{rodriguez2012caging}.

Broadly, we can distinguish between control and planning for manipulation tasks. Planning methods take the full plan of action into account, generally simulating the whole process offline to check for conditions of feasibility and optimality~\cite{kavraki2016motion} of the motion. Control-oriented approaches tend to be real-time, sensor feedback-based methods that aid in the performance of the motion at an implementation level~\cite{dafle2014extrinsic}, generally ensuring local optimality. Task Frame representational approaches~\cite{prats2010framework}, and methods involving friction and physics modeling~\cite{chavan2015prehensile} tend to fall into this category.

This project is aimed at developing a manipulation planner for a robot arm that uses the motion primitives of grasping, caging, and pushing, along with arm translation and rotation. Thus far, most prior work on planning has assumed the grasping primitive to be true, thus reducing the problem to finding a path in $C$--space of the robot. One such set of algorithms distinguishes between \textit{transit} paths, where the object is not in contact with the robot, and \textit{transfer} paths, where the contact is achieved~\cite{alami1994two}. In the interim of switching between these two modes, it is assumed that an immobilizing grasp can be found. Work specific to prehensile motions (caging, pushing) includes efforts to geometrically model a pushing motion to facilitate a grasp~\cite{dogar2010push}, and caging-specific tree-search based algorithms~\cite{diankov2008manipulation}. There have also been efforts to incorporate object physics, friction and mechanics modeling for highly dynamic motion plans that allow for the object to move as an independent object during the plan~\cite{dogar2012physics}. Finally, there are task-space based planners that assume a quasi-static motion of objects, while taking into account the uncertainty in their position, implemented using an RRT-based search~\cite{berenson2009addressing}.

I aim to extend the state of the art by generating a planner that can switch between pushing, caging and grasping based on the object, environment and goal. At present, tree-search and probabilistic roadmap~\cite{kavraki2016motion} based approaches are being explored where the switching condition emerges as a consequence of a suitable search heuristic cost function.


\section{Technical Contribution}

\bibliographystyle{splncs03}
\bibliography{IEEEabrv,references}


\end{document}
